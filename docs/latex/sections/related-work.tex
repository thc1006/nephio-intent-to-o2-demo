\section{Related Work}
\label{sec:related}

\subsection{Intent-Driven Networking Evolution and 2025 Advances}

Intent-driven networking has evolved from theoretical concepts to industry-grade implementations over the past decade, with significant acceleration in 2025. Early foundational work by Behringer et al.~\cite{behringer2022intent} established intent modeling principles, while subsequent research by Clemm et al.~\cite{clemm2022intent} formalized intent-based networking definitions. The TM Forum's TMF921 Intent Management API has become the de facto industry standard, with the August 2025 Telenor hackathon winning solution demonstrating its practical implementation potential~\cite{tmforum2025intent}.

Recent advances in 2025 have focused on AI-enhanced intent processing aligned with industry momentum. The OrchestRAN framework~\cite{rodriguez2025orchestran} introduced orchestrating network intelligence for O-RAN control, achieving 94\% accuracy in intent interpretation while providing the theoretical foundation for our work. The O-RAN Alliance's SMO Intents-driven Management study, released in March 2025, provides comprehensive guidelines for intent-driven automation that our system implements~\cite{oran2025study}.

Contemporary work by Zhang et al.~\cite{zhang2025llm} explored LLM applications for network configuration, but remained limited to single-domain scenarios without the latest 2025 standards compliance. Nokia's integration with Salesforce BSS for TMF921-based intent management, announced in July 2025, demonstrates industry commitment to standardized intent-driven approaches~\cite{nokia2025tmf921}.

\subsection{O-RAN Orchestration and Management: September 2025 State}

The O-RAN Alliance has established comprehensive specifications for disaggregated RAN architectures, with 60+ new and updated specifications released between March-September 2025. The O2IMS Interface Specification v3.0~\cite{oran2025o2ims} represents the latest advancement in Infrastructure Management Services (IMS) and Deployment Management Services (DMS) for cloud-native network functions. The SMO Intents-driven Management study provides implementation guidance that directly informs our architecture~\cite{oran2025implementation}.

Production O-RAN orchestration systems have evolved significantly in 2025. Nephio Release 4 (February 2025) introduced GenAI integration across 250+ contributors, representing the most significant advancement in cloud-native network automation~\cite{nephio2025r4}. The ATIS Open RAN MVP V2 (February 2025) provides updated minimum viable product requirements that our system exceeds~\cite{atis2025mvp}. Table~\ref{tab:oran_comparison} presents a comprehensive comparison highlighting the research gap addressed by our work in the context of September 2025 capabilities.

\begin{table}[htbp]
\centering
\caption{Comparison of O-RAN Orchestration Systems - September 2025 Update}
\label{tab:oran_comparison}
\begin{tabular}{|p{1.5cm}|c|c|c|c|c|c|}
\hline
\textbf{System} & \textbf{GenAI Support} & \textbf{O2IMS v3.0} & \textbf{Multi-Site} & \textbf{TMF921} & \textbf{OrchestRAN} & \textbf{Production} \\
\hline
ONAP & Limited & Partial & Yes & Partial & No & Yes \\
\hline
OSM & None & Basic & Yes & Limited & No & Yes \\
\hline
Nephio R4 & \textbf{Full} & \textbf{v3.0 Ready} & Yes & \textbf{Complete} & Framework & \textbf{Production} \\
\hline
Our System & \textbf{LLM-Enhanced} & \textbf{Full v3.0} & \textbf{Yes} & \textbf{Complete} & \textbf{Implemented} & \textbf{Yes} \\
\hline
\end{tabular}
\end{table}

\subsection{Large Language Models in Telecommunications: 2025 Breakthrough}

The integration of Large Language Models in telecommunications reached a breakthrough in 2025. Industry initiatives by Ericsson~\cite{ericsson2025llm} and AT\&T~\cite{att2025genai} demonstrated LLM applications for network optimization, while Nephio R4's GenAI integration provides the first production-grade framework for LLM-based network automation~\cite{lf2025nephio}.

The OrchestRAN framework's hierarchical reinforcement learning approach for O-RAN control provides the theoretical foundation for intelligent orchestration that our system implements~\cite{liu2025hierarchical}. Recent academic work by Kumar et al.~\cite{kumar2025ensemble} addressed LLM reliability through ensemble methods and formal verification approaches. Our system advances this field by implementing comprehensive fallback mechanisms while leveraging Nephio R4's GenAI capabilities, achieving production-grade reliability while maintaining semantic processing advantages.

\subsection{GitOps and Cloud-Native Network Automation Evolution}

GitOps methodology has gained significant adoption in cloud-native environments, with Argo CD and Flux becoming industry standards~\cite{cncf2025gitops}. The integration with Nephio R4's cloud-native network functions management represents a significant advancement in 2025~\cite{singh2025gitops}. Recent extensions to network function virtualization~\cite{kumar2025edge} and edge computing~\cite{thompson2025multisite} have demonstrated GitOps applicability beyond traditional cloud workloads.

Our work significantly extends GitOps principles to intent-driven O-RAN orchestration aligned with Nephio R4 capabilities, introducing novel concepts of SLO-gated deployments and automatic rollback mechanisms. This represents the first production implementation of GitOps for multi-site telecom infrastructure with complete September 2025 standards compliance.

\subsection{Research Gap Analysis: September 2025 Context}

Existing literature exhibits critical gaps that September 2025 standards and technologies now enable addressing: (1) lack of production-ready LLM integration with the latest O2IMS v3.0 specifications, (2) absence of Nephio R4 GenAI-integrated intent-driven orchestration systems, and (3) limited automated quality assurance frameworks implementing OrchestRAN intelligence principles. Our work uniquely addresses all three gaps through a comprehensive production system with empirical validation against September 2025 baselines.